Em Pelas Ondas do Rádio: Cultura Popular, Camponeses e o MEB analisa a participação de camponeses do nordeste brasileiro no Movimento de Educação de Base. A perspectiva da tese é a de demonstrar como os trabalhadores envolvidos com as escolas radiofônicas elaboraram ações para manutenção e reprodução da escola em sua comunidade, visando obter os benefícios necessários à reprodução e melhoria de seu modo de vida. A partir de representações políticas e culturais singulares, dentre as quais vigoraram: um sentido para escola, um papel para o sindicato e para participação política, preceitos do direito de uso da terra e dos direitos do trabalho, assim como, sentidos múltiplos para o uso do rádio como meio de comunicação, informação e lazer, os camponeses do MEB, foram coadjuvantes da proposição católica modernizadora de inícios de 1960. Isto posto, demarca que a ação do camponês nordestino e seu engajamento político, seja no MEB, nos sindicatos rurais, nas Juventudes Agrárias Católicas (JAC’s), no MCP, e nas mais diversas instâncias dos movimentos sociais do período, não se apartaram do processo modernizador. Neste sentido, considera-se que a modernização brasileira foi pauta das instituições, organismos políticos e partidos, assim como, do movimento social, instância em que ela foi ressignificada a partir de elementos da vida material, que envolviam diretamente, no momento em questão, a problemática do direito a terra, do direito a educação e cultura e dos direitos do trabalho.

% Separe as palavras-chave por ponto
\palavraschave{Camponeses. Cultura popular. Educação de adultos. Escola rural.}